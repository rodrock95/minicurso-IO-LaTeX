\documentclass[12pt, a4paper, oneside]{book}

\usepackage[top = 2cm, bottom = 2cm, left = 2.5cm, right = 2.5cm]{geometry}
\usepackage[T1]{fontenc}
\usepackage[brazilian]{babel} %Pacote de idiomas
\usepackage[utf8x]{inputenc}
\usepackage{setspace}
\usepackage[normalem]{ulem}

\onehalfspacing

\begin{document}

%Cabeçalho
\title{Minicurso LaTeX}
\author{Rodrigo \footnote{Graduado em ADS}}
\date{30 de Setembro de 2022}
\maketitle

\hrulefill

\begin{center}
\Large{\textit{DOCUMENTOS E TEXTOS}}
\end{center}

\hrulefill

\vspace{0.5cm}

%JUSTIFICADO
Trabalho final de graduação do curso de \underline{Análise e desenvolvimento de sistemas} da Faculdade de tecnologia do Estado de São Paulo apresentado como pré requisito para a obtenção do grau de tecnólogo em Análise e desenvolvimento de sistemas. Trabalho final de graduação do curso de Análise e desenvolvimento de sistemas da Faculdade de tecnologia do Estado de São Paulo apresentado como pré \textit{requisito} para a obtenção do grau de tecnólogo em Análise e desenvolvimento de sistemas. Trabalho final de graduação do curso de Análise e desenvolvimento de sistemas da Faculdade de tecnologia do Estado de São Paulo apresentado como pré requisito para a obtenção do grau de tecnólogo em Análise e desenvolvimento de sistemas \footnote{Nota de rodapé}.

%ALINHADO À ESQUERDA
\begin{flushleft}
Trabalho \sout{final} de graduação do curso de \textit{\textbf{Análise e desenvolvimento de sistemas}} da Faculdade de tecnologia do Estado de São Paulo apresentado como pré requisito para a obtenção do grau de tecnólogo em Análise e desenvolvimento de sistemas. Trabalho final de graduação do curso de Análise e desenvolvimento de sistemas da Faculdade de tecnologia do Estado de São Paulo apresentado como pré requisito para a obtenção do grau de tecnólogo em Análise e desenvolvimento de sistemas.Trabalho final de graduação do curso de Análise e desenvolvimento de sistemas da Faculdade de tecnologia do Estado de São Paulo apresentado como pré requisito para a obtenção do grau de tecnólogo em Análise e desenvolvimento de sistemas.
\end{flushleft}

%CENTRALIZADO
\begin{center}
Trabalho final de graduação do curso de Análise e desenvolvimento de sistemas da Faculdade de tecnologia do Estado de São Paulo apresentado como pré requisito para a obtenção do grau de tecnólogo em Análise e desenvolvimento de sistemas. Trabalho final de graduação do curso de Análise e desenvolvimento de sistemas da Faculdade de tecnologia do Estado de São Paulo apresentado como pré requisito para a obtenção do grau de tecnólogo em Análise e desenvolvimento de sistemas.Trabalho final de graduação do curso de Análise e desenvolvimento de sistemas da Faculdade de tecnologia do Estado de São Paulo apresentado como pré requisito para a obtenção do grau de tecnólogo em Análise e desenvolvimento de sistemas.\\
\end{center}

"Lorem ipsum dolor sit amet, consectetur adipiscing elit, sed do eiusmod tempor incididunt ut labore et dolore magna aliqua. Ut enim ad minim veniam, quis nostrud exercitation ullamco laboris nisi ut aliquip ex ea commodo consequat. Duis aute irure dolor in reprehenderit in voluptate velit esse cillum dolore eu fugiat nulla pariatur. Excepteur sint occaecat cupidatat non proident, sunt in culpa qui officia deserunt mollit anim id est laborum."

"Lorem ipsum dolor sit amet, consectetur adipiscing elit, sed do eiusmod tempor incididunt ut labore et dolore magna aliqua. Ut enim ad minim veniam, quis nostrud exercitation ullamco laboris nisi ut aliquip ex ea commodo consequat. Duis aute irure dolor in reprehenderit in voluptate velit esse cillum dolore eu fugiat nulla pariatur. Excepteur sint occaecat cupidatat non proident, sunt in culpa qui officia deserunt mollit anim id est laborum."

\textit{Listagem sem numeração abaixo:}

\begin{itemize}
	\item[$\heartsuit$] Qualquer coisa
	\item[$\partial$] Olá, estou aqui!
\end{itemize}

\textit{Listagem com numeração abaixo:}

\begin{enumerate}
	\item Qualquer coisa
	\item beleza
\end{enumerate}

\chapter{Introdução}
\label{cap1}

Trabalho final de graduação do curso de Análise e desenvolvimento de sistemas da Faculdade de tecnologia do Estado de São Paulo apresentado como pré requisito para a obtenção do grau de tecnólogo em Análise e desenvolvimento de sistemas. Trabalho final de graduação do curso de Análise e desenvolvimento de sistemas da Faculdade de tecnologia do Estado de São Paulo apresentado como pré requisito para a obtenção do grau de tecnólogo em Análise e desenvolvimento de sistemas.Trabalho final de graduação do curso de Análise e desenvolvimento de sistemas da Faculdade de tecnologia do Estado de São Paulo apresentado como pré requisito para a obtenção do grau de tecnólogo em Análise e desenvolvimento de sistemas.

\hrulefill
\begin{center}
\Large{Página destinada à assinaturas}
\end{center}

\hrulefill

\vspace{2cm}

\begin{center}
\rule{10cm}{0.02cm}\\
Rodrigo Camara Barboza
\end{center}

\vspace{1cm}

\begin{center}
\rule{10cm}{0.02cm}\\
Carlos Molina
\end{center}
\newpage

\hrulefill

\begin{center}
\Large{\textit{AMBIENTE MATEMÁTICO}}
\end{center}

\hrulefill

\vspace{0.5cm}

Segundo a equação $ x = 2 $, $x$ está valendo 2.

\begin{equation}
x = 2.
\end{equation}

\begin{equation}
\left\lbrace \frac{5}{2} \right\rbrace
\end{equation}

O valor de $\sin 60º$ é:

\begin{equation}
\frac{\sqrt{3}}{2}
\end{equation}



\end{document}